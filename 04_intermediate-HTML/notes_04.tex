\documentclass{article}
\usepackage{graphicx}
\usepackage[T1]{fontenc}

\begin{document}
	
	\title{Intermediate HTML}
	\author{Nico Limacher}
	
\maketitle
\section{Objectives}

\begin{enumerate}
	\itemsep0em
	\item different elements in HTML tables.
	\item making HTML tables.
	\item writing valid HTML forms.
\end{enumerate}

\section{HTML Tables}
	\subsection{Option 1:}
	\begin{verbatim}
	<table>
	  <tr>
	    <td>Red</td>
	    <td>Orange</td>
	  </tr>
	  <tr>
	    <td>Green</td>
	    <td>Blue</td>
	  </tr>
	</table>
	\end{verbatim}
	Becomes:
	
	\medskip
	\begin{tabular}{| l | l |}
		\hline
		Red & Orange \\ \hline
		Green & Blue\\ \hline
	\end{tabular}
	\medskip
	
	Note: No border will be shown without CSS
	
	\subsubsection{Adding Header Row}
	\begin{verbatim}
	<table>
	  <tr>
	    <th>Name</th>
	    <th>Age</th>
	  </tr>
	  <tr>
	    <td>Bert</td>
	    <td>5</td>
	  </tr>
	</table>
	\end{verbatim}
	Becomes:
	
	\medskip
	\begin{tabular}{| l | l |}
		\hline
		\textbf{Name} & \textbf{Age} \\ \hline
		Bert & 5 \\ \hline
	\end{tabular}
	\medskip
	
	\subsection{Option 2:}
	To distinguish headers from normal rows: use <thead> and <tbody>
	\begin{verbatim}
	<table>
	  <thead>
	    <th>Name</th>
	    <th>Age</th>
	  </thead>
	  <tbody>
	    <td>Bert</td>
	    <td>5</td>
	  </tbody>
	</table>
	\end{verbatim}
	
	Looks exactly the same, but more readable

\section{Introduction To Forms}
	Getting user input

	\begin{verbatim}
	<form action="index.html" method="post">
	  <!-- all our inputs go here -->
	</form>
	\end{verbatim}
	
	\paragraph*{action}
	Where to send data
	\paragraph*{method}
	The type of HTTP request. Can be "post", "get"...
	
	\subsection{Inputs}
	\begin{verbatim}
		<input type="text">
		<input type="date">
		<input type="color">
		<input type="checkbox">
	\end{verbatim}

\section{The Form Tag}
\paragraph*{Form}
Inputs grouped together

\section{Labels}
Making our site accessible (by text-to-speech). a good practice

\subsection{Two Options}
\begin{enumerate}
	\item Placing everything within <label> tags:
	\begin{verbatim}
	<label>
	  Username:
	  <input type="text" placeholder="username">
	</label>
	\end{verbatim}
	\item For and id pairings
	\begin{verbatim}
	<label for="username">Username:</label>
	<input id="username" type="text" placeholder="username">
	\end{verbatim}
\end{enumerate}

\section{Form Validations}
Simple validations with plain HTML
\begin{enumerate}
	\item "required" attribute. If present then field cannot be blank
	\begin{verbatim}
	<input name="username" type="text" placeholder="username" required>
	\end{verbatim}
	\item Format of data
	\begin{itemize}
		\item Change "type" attribute to take advantage of built in validation:
		\begin{verbatim}
		<input name="username" type="email" placeholder="email" required>
		\end{verbatim}
		\item Note: error messaging handled by browser, no control on our end
	\end{itemize}
\end{enumerate}
\section{Drop-downs and Radio Buttons}
\begin{enumerate}
	\item Radio Buttons
	
	Another type of input tag. Cannot be toggled off once selected whereas a checkbox can
	\begin{center}
		Setting up connected radio buttons using Name:
	\end{center}
	\begin{verbatim}
	<label for="dogs">Dogs:</label>
	<input name="petChoice" id="dogs" type="radio" value="dog">
	
	<label for="cats">Cats:</label>
	<input name="petChoice" id="cats" type="radio" value="cat">
	\end{verbatim}
	Note: "value" determines what is sent in the query once submitted
	\item Select
	
	Creates drop-down menus
	\begin{verbatim}
	<select name="color">
	  <option>Red</option>
	  <option>Blue</option>
	  <option>Yellow</option>
	</select>
	\end{verbatim}
	\begin{itemize}
		\item "name" determines the key of the query
		\item either "value" attribute of option tag or the actual value of tag is passed in key-value pair of query
	\end{itemize}

	\item Text Area
	\begin{verbatim}
	<textarea name="bio" rows="10" cols="10"></textarea>
	\end{verbatim}
	
	Multiple line long text areas\\Rows and columns give dimensions
	
	\begin{verbatim}
	
	\end{verbatim}
\end{enumerate}

\end{document}