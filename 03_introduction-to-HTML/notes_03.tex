\documentclass{article}
\usepackage{graphicx}
\usepackage[T1]{fontenc}

\begin{document}
	
	\title{03 - Introduction to HTML}
	\author{Nico Limacher}
	
\maketitle


\section{Objectives}
\begin{enumerate}
	\itemsep0em
	\item Write simple HTML docs
	\item Difference between closing and self-closing tags
	\item Tags with attributes
	\item Using MDN as a reference
	\item Given a website, reproduce
\end{enumerate}

\section{Boilerplate and Comments}
\subsection{General Structure of Tags}
\begin{verbatim}
<tagName> ...some content... </tagName>
\end{verbatim}

\subsection{Best Resources}
\begin{itemize}
	\itemsep0em
	\item MDN - Mozilla Developer Network
	\item Reading assignment - MDN intro to HTML
\end{itemize}

\subsection{Commenting in HTML}
use <! - - and - ->

\subsection{Boilerplate}
\begin{verbatim}
<!DOCTYPE html>
<html lang="en" dir="ltr">
  <head>
    <meta charset="utf-8">
    <title>IN HERE IS THE TEXT IN THE BROWSER TAB</title>
  </head>
  <body>
  </body>
</html>
\end{verbatim}

\section{Basic Tags}
\subsection{Frequently Used Tags}
\begin{center}
	\begin{tabular}{| l | l | p{10cm} |}
		\hline
		\textbf{Tag} & \textbf{Block/In-line} & \textbf{Explanation} \\ \hline
		H1 & Block & Header of Size 1 \\ \hline
		H2 & Block & Header of Size 2 \\ \hline
		H3 & Block & Header of Size 3 \\ \hline
		H4 & Block & Header of Size 4 \\ \hline
		H5 & Block & Header of Size 5 \\ \hline
		H6 & Block & Header of Size 6 \\ \hline
		p & Block & Creates self-contained paragraph \\ \hline
		strong & In-line & Bolds the following text \\ \hline
		em & In-line & Italicizes the following text \\ \hline
		img & block? & Add an image \\ \hline
		a & In-line & Add a link. Note: be careful to specify local or http \\ \hline
	\end{tabular}
\end{center}

\subsection{Block vs. Inline}
\subsubsection{Block}
Will automatically start the element on a new line.
\subsubsection{In-line}
Can be used within other tags without automatically new-lining.

\section{HTML lists}
\subsection{Ordered List}
\begin{verbatim}
<ol>
  <li>elementary school</li>
  <li>middle school</li>
  <li>high school</li>
</ol>
\end{verbatim}
Becomes:
\begin{enumerate}
	\itemsep0em
	\item elementary school
	\item middle school
	\item high school
\end{enumerate}
\subsection{Unordered List}
\begin{verbatim}
<ul>
  <li>fish</li>
  <li>cat</li>
  <li>dog</li>
</ul>
\end{verbatim}
Becomes:
\begin{itemize}
	\itemsep0em
	\item fish
	\item cat
	\item dog
\end{itemize}

\section{Divs and Spans}
A way to group elements for formatting purposes.
\subsection{Div}
Group multiple elements together into a box or some other graphical entity. For example, grouping a title and paragraph into a different colored box.
\subsection{Span}
Also a generic container but an in-line version, rather than a block level one.

\section{Attributes}
Adding additional information
\begin{verbatim}
The key-value pair
<tag name="value">...</tag>
\end{verbatim}

\begin{center}
	\begin{tabular}{| l | p{10cm} |}
		\hline
		\textbf{Attr.} & \textbf{Explanation} \\ \hline
		src & name or URL link of image file \\ \hline
		href & link for an <a> tag \\ \hline
	\end{tabular}
\end{center}


\end{document}