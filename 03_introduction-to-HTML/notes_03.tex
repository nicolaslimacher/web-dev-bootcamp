\documentclass{article}
\usepackage{graphicx}
\usepackage[T1]{fontenc}

\begin{document}
	
	\title{03 - Introduction to HTML}
	\author{Nico Limacher}
	
\maketitle
\section{Boilerplate and Comments}
\subsection{Objectives}

\begin{enumerate}
	\itemsep0em
	\item write simple HTML docs
	\item difference between closing/ self closing tags
	\item tags with attributes
	\item using MDN as a reference
	\item given a website, reproduce
\end{enumerate}

\subsection{General Structure of Tags}
<tagName> some content </tagName>

\subsection{Best Resources}
\begin{itemize}
	\itemsep0em
	\item MDN - mozilla developer network
	\item reading assignment - MDN intro to html
\end{itemize}

\subsection{Commenting in HTML}
use <! - - and - ->

\subsection{Boilerplate}
\begin{verbatim}
<!DOCTYPE html>
<html lang="en" dir="ltr">
  <head>
    <meta charset="utf-8">
    <title>IN HERE IS THE TEXT IN THE BROWSER TAB</title>
  </head>
  <body>
  </body>
</html>
\end{verbatim}

\section{Basic Tags}
\subsection{Frequently Used Tags}
\begin{center}
	\begin{tabular}{| l | l | p{10cm} |}
		\hline
		\textbf{Tag} & \textbf{Block/In-line} & \textbf{Explanation} \\ \hline
		H1 & Block & Header of Size 1 \\ \hline
		H2 & Block & Header of Size 2 \\ \hline
		H3 & Block & Header of Size 3 \\ \hline
		H4 & Block & Header of Size 4 \\ \hline
		H5 & Block & Header of Size 5 \\ \hline
		H6 & Block & Header of Size 6 \\ \hline
		p & Block & Creates self-contained paragraph \\ \hline
		strong & In-line & Bolds the following text \\ \hline
		em & In-line & Italicizes the following text \\ \hline
	\end{tabular}
\end{center}

\subsection{Block vs. Inline}
\begin{center}
BLOCK - Takes its own line, always

IN-LINE - Can be used within other tags without automatically newlining
\end{center}

\end{document}